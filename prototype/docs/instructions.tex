\documentclass[a4paper,11pt]{article}
% Fontsizes: normalsize, small, footnotesize, scriptsize, tiny

%% ++ Language ++

\usepackage{polyglossia}
\setdefaultlanguage[variant=british]{english}

%% ++ Layout ++

\frenchspacing
% \usepackage{setspace}\onehalfspacing
\usepackage[a4paper,margin=3cm]{geometry}
\setlength{\footskip}{1cm}

% To move the title up and reduce space between the date and the first
% paragraph (does not work with KOMA scrartcl),
\usepackage{titling}
\setlength{\droptitle}{-5\baselineskip}
% \predate{\begin{center}\large}
\postdate{\par\end{center}\vspace{-3\baselineskip}}

% To decrease space between sections (does not work with KOMA scrartcl),
% \usepackage[compact]{titlesec} 

%% ++ Figures ++

\usepackage{graphicx}
\usepackage{svg}
\usepackage[off]{svg-extract}
\svgsetup{clean=true,inkscapearea=page,inkscapepath=svg-inkscape/}
\graphicspath{{img/}}
\svgpath{{svg/}}

%% ++ Captions ++

\usepackage{caption}
\captionsetup{format=plain, font={small,singlespacing},
  labelfont=bf, labelsep=period}
    %labelsep=newline, justification=RaggedRight}

%% ++ Lists ++

\usepackage{enumitem}
\setlist{noitemsep}

% ++ Fonts

\usepackage{fontspec}
\usepackage{unicode-math}
\unimathsetup{math-style=ISO}%,warnings-off={mathtools-colon}}
\defaultfontfeatures{Mapping=tex-text}
\setmainfont{TeX Gyre Heros}
%\setmathfont{TeX Gyre Schola Math}[Scale=MatchLowercase]
\setsansfont{TeX Gyre Heros}[Scale=MatchLowercase]
\setmonofont{TeX Gyre Cursor}[Scale=MatchLowercase]
\usepackage{relsize}

%% ++ Headers/footers

\usepackage{fancyhdr}
\usepackage{scrdate}
\pagestyle{fancy}
\renewcommand{\headrulewidth}{0pt}
\fancyhf{}
\fancyfoot[L]{\tiny Antonio González, Ver.~\ISOToday}
\fancyfoot[R]{\thepage}


%% ++ Optional extra formatting

% Remove sections numbering
\setcounter{secnumdepth}{-1}

% Maths. \begin{equation} or \begin{align}
% \usepackage{mathtools}

%% ++ Hyperref
\usepackage{hyperref}
\hypersetup{
  pdfauthor={JA González},
  pdftitle={Lick sensor},
  pdfsubject={},
  hidelinks
}
\urlstyle{same}
% \newcommand\email[1]{\href{mailto:#1}{#1}}


%% ++ Begin
\title{\textbf{Lick sensor}}
%\author{JA González}
\date{}

\begin{document}
\maketitle
\thispagestyle{fancy}
% \vskip-4em

\begin{figure}[h]
    % \centering
    \includesvg[pretex=\footnotesize,width=\columnwidth]{board}
    \caption{Overview of the lick-sensing device.}\label{fig:device}
    \end{figure}

\section{Wiring}
\begin{enumerate}
    \item Use a cable to connect input ``0'' (second from left to right
    on the ``Lick sensor connector'', see Fig.~\ref{fig:device}) to the
    (metal) spout of a drinking bottle (Fig.~\ref{fig:shielding}A).
    \item Connect ``Digital out'' to your data acquisition system with a
    BNC cable.
    \item (Optional) Connect ``Analog out'' to your data acquisition
    system with a BNC cable.
    \item Connect a USB cable to ``USB''. This powers the device. Do
    this only after all other cables have been connected and the
    drinking bottle is in place.
\end{enumerate}

\section{Operation}
\begin{enumerate}
    \item Record data from ``Digital out''. This will be the licking
    signal: a step will appear with every lick
    (Fig.~\ref{fig:shielding}B).
    \item (Optional) Record the signal from ``Analog out''. This signal
    increases or decreases with changes in capacitance in the touch
    sensor and it could be useful at some point if we needed to
    troubleshoot the system.
\end{enumerate}

\section{Adjusting settings}

The default settings should work just fine. If some tuning is needed,
each setting on the display can be adjusted using the buttons:

\begin{itemize}
    \item ``Previous'' and ``Next'' take you through the different
        possible settings.
    \item ``+'' and ``−'' increase or decrease the value of the setting
        currently on display.
\end{itemize}

The most relevant settings are perhaps ``Touch threshold'' and ``Release
threshold'': when the analogue signal goes above Touch threshold, a
touch event (i.e.\ a lick) is triggered. When this signal goes below
Release threshold, the event (lick) is considered to have
ended.\footnote{Note that the value of Touch threshold must always be
greater than that of Release threshold.} Thus, these two values set the
sensitivity of the system and introduce hysteresis to limit jitter.

% Increase Touch threshold if the system is too sensitive.
% Increase Release threshold if 


\section{Optional: grounding}

It may be that animals trigger false positive lick events when they
touch with their paw the drinking bottle. These false positive events
may be reduced by shielding the spout of the water bottle with foil
(Fig.~\ref{fig:shielding}C,D) and connecting this foil to ``G'' (ground)
in the Lick sensor connector. Just ensure that signal and ground cables
do not make contact with each other.


\begin{figure}
    \centering
    \includesvg[pretex=\scriptsize,width=\columnwidth]{lickometer-system}
    \caption{(\textbf{A}) A cable connecting the spout of the drinking
    bottle to the lick sensor is all the wiring required at the animal's
    end. (\textbf{B}) The output of the lick sensor (``Digital out'' in
    Fig.~\ref{fig:device}) is a series of voltage steps, each indicating
    a lick event. (\textbf{C}) Optionally, shielding the spout of the
    drinking bottle may help reduce false positives that arise when
    animals hold the spout while drinking. To do this, surround the
    spout with a piece of conductive material (e.g.\ foil) and connect
    this to ground (``G'') on the Lick sensor connector. Use heat shrink
    or similar to ensure that the foil (ground) and the spout (sensor)
    do not touch each other. The tip of the spout should be clear of the
    foil shielding. (\textbf{D}) Keeping ground and signal cables
    separate, connect these the to the lick
    sensor.}\label{fig:shielding}
\end{figure}


\end{document}
